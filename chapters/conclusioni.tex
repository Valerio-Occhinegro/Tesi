\chapter{Conclusioni}
\label{cap:conclusioni}

\section{Raggiungimento degli obiettivi}
Per terminare il percorso di stage è necessario verificare il grado di soddisfacimento degli obiettivi preimposti, tale processo viene rappresentato nella tabella \ref{soddisfazione}.
Come è possibile notare, tutti i requisiti obbligatori sono stati soddisfatti, ma c'è ancora spazio per dei possibili miglioramenti.

\begin{table}[!htbp]
    \centering
    \begin{tabularx}{0.8\textwidth}{|c|X|X|}
    \hline
    \textbf{Codice} & \textbf{Descrizione} & \textbf{Soddisfacimento}\\
    \hline
    O01 & Implementare un sistema robusto per la cattura degli screenshot, assicurando l’integrazione con il database per l’archiviazione e l’analisi. & Soddisfatto\\
    \hline
    O02 & Garantire la creazione di una documentazione tecnica completa che supporti sia l’uso che la manutenzione  del sistema sviluppato. & Soddisfatto\\
    \hline
    O03 & Creazione di un IA in grado di suddividere in cluster i siti web. & Soddisfatto\\
    \hline
    O04 & Creazione di un IA classificativa in grado di assegnare un punteggio ai siti-web analizzati. & Soddisfatto\\
    \hline
    D01 & Aggiungere la valutazione creata dall'IA nel database utilizzato dal CRM. & Soddisfatto\\
    \hline
    D02 & Automatizzare l'invio di e-mail alle aziende che hanno ottenuto una valutazione scarsa. & Soddisfatto\\
    \hline
    F01 & Migliorare la raccolta dei dati delle aziende dal web & Non soddisfatto\\
    \hline
    F02 & Aggiornare automaticamente la valutazione dei siti web & Soddisfatto\\
    \hline
    \end{tabularx}
    \caption{Grado di soddisfacimento degli obiettivi}
    \label{soddisfazione}
\end{table}

\subsection{Clustering}
I risultati di clustering migliori sono stati ottenuti con l'estrazione delle feature tramite ResNet50, ma il processo necessita di grossi miglioramenti per essere funzionale;
infatti visualizzando manualmente i cluster ottenuti non è possibile notare delle distinzioni nette tra immagini appartenenti a una o all'altra categoria.
Perciò è necessario proseguire con gli studi comprendendo più a fondo come utilizzare l'algoritmo K-means al meglio e testando altri algoritmi di clustering.
Il lavoro svolto può essere considerato un'ottima base di partenza poiché si è concentrato sullo studio preliminare di varie metodologie; arrivando a escludere quelle meno adatte al compito di estrazione delle feature.

\subsection{Valutazione delle immagini}
Per quanto riguarda la fase di valutazione, i risultati ottenuti dal fine-tuning di ResNet50 sembrano essere ottimali, ottenendo un'accuratezza del 92\% e l'assegnazione di voti congrui a quelli che lo sviluppatore si aspetterebbe di visualizzare.
Il prodotto è comunque migliorabile aggiungendo una quantità più elevata di screenshot al dataset e di conseguenza aumentando il range di siti a cui adattarsi.

\section{Valutazione personale}
L'esperienza di stage è stata di grande impatto, ha consentito di sfruttare le conoscenze acquisite e ha messo in risalto numerose lacune da colmare.
Il progetto mi ha consentito di apprendere nuove conoscenze nell'ambito del machine learning, argomento che personalmente non avevo mai approfondito. 
Le scarse conoscenze in ambito IA hanno comportato una grossa sfida, che una volta superata ha portato a un alto livello di soddisfazione.

