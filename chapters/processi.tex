\chapter{Processi e metodologie}
\label{cap:processi-metodologie}

In questo capitolo sono riassunti i processi e le metodologie che lo studente ha applicato durante lo sviluppo del progetto di stage. 

\section{Gestione della configurazione}
La gestione della configurazione è il processo tramite il quale è possibile identificare, controllare e coordinare i vari componenti del software e le risorse ad esso associate durante l'intero ciclo di vita del prodotto. Grazie a questo processo gli sviluppatori possono tenere traccia delle modifiche e gestire le varie versioni garantendo il funzionamento del prodotto nel tempo.

\subsection{Tecnologie di supporto}
Le tencologie utilizzate per il versionamento del prodotto sono:
\begin{itemize}
    \item Git: è un DVCS(Distributed Version Controll System) ampiamente diffuso che consente di gestire e monitorare i cambiamenti del codice e della documentazione. Questo strumento è fondamentale per avere una storia completa dello sviluppo da poter sfruttare in caso di necessità. Il codice viene inserito all'interno di un repository, ossia una cartella che contiene tutti i file relativi al progetto, dove è possibile salvare ogni cambiamento tramite commit. Il commit segnala ogni modifica e la data in cui viene attuata, è dunque sufficiente spostarsi tra i vari commit per tornare ad una versione più o meno aggiornata.    
    \item Gitea: è un servizio che supporta le repository di Git, è simile a GitHub, Bitbucket e GitLab, ma ha il vantaggio di essere self-hosted. 
    Tutti i repository aziendali sono mantenuti sul server interno, ciò consente di avere una risposta molto rapida.

    commit e branch tutti i salcazzi vari
\end{itemize}

\section{Processo sviluppo prodotto}

%processo di sviluppo ci schiaffo lo scrum

%processi di supporto tutte le menate di documentazione ecc

%processi organizzativi, its, repo, formazione