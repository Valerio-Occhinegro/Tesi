\chapter{Il progetto}
\label{cap:descrizione-stage}

\intro{Questo capitolo contiene le informazioni preliminari che sono state fornite allo studente e le misure preventivie adottate per iniziare la produzione.}\\

\section{Analisi del progetto}

Il progetto si basa sull'integrazione di un nuovo \emph{\gls{workflow}}\glsfirstoccur all'interno di un applicativo preesistente; lo studente deve occuparsi dello sviluppo di una soluzione in grado di migliorare la portata dell'azienda per quanto riguarda il raggiungimento di nuovi clienti.

\subsection{SalesCRM}
SalesCRM è il \gls{CRM} che viene utilizzato dai commerciali interni all'azienda; contiene tutte le informazioni relative ai contatti che sono stati raggiunti dal reparto vendite. 
Gli utenti sono in grado di immagazzinare i dati raccolti su clienti e potenziali clienti tramite un interfaccia intuitiva sviluppata utilizzando \nameref{tec:Laravel} e \nameref{tec:Filament}.

\subsubsection{Componenti principali}
\begin{itemize}
    \item \emph{Homepage} (Fig.~\ref{fig:salesCRM-home}) contenente grafici informativi.
    
    \begin{figure}[!h] 
        \centering 
        \includegraphics[width=0.8\columnwidth]{progetto/Home_page.png} 
        \caption{Homepage di SalesCRM}
        \label{fig:salesCRM-home}
      \end{figure}

    \item Il modulo contenente il calendario (Fig.~\ref{fig:salesCRM-calendario}) è utile per pianificare appuntamenti e visite.
    
    \begin{figure}[!h] 
        \centering 
        \includegraphics[width=0.8\columnwidth]{progetto/Calendario.png} 
        \caption{Calendario contenuto all'interno di SalesCRM}
        \label{fig:salesCRM-calendario}
      \end{figure}

    \item La sezione della mappa (Fig.~\ref{fig:salesCRM-mappa}) viene sfruttato per conoscere il posizionamento dei clienti e organizzare incontri locali.
    
    \begin{figure}[!h] 
        \centering 
        \includegraphics[width=0.8\columnwidth]{progetto/Mappa.png} 
        \caption{Mappa contenuta all'interno di SalesCRM}
        \label{fig:salesCRM-mappa}
      \end{figure}
    
\end{itemize}

\newpage

\subsection{Integrazione}
\begin{itemize}
    \item Raccolta di dati: l'applicativo ha una funzionalità di \emph{\gls{web scraping}}\glsfirstoccur che raccoglie i link dei siti web di tutti i potenziali clienti, il compito del tirocinante è quello di creare un \emph{\gls{workflow}} automatico che si integri con il \emph{web scraper} per acquisire screenshot delle varie pagine del sito web. 
    \item Clusterizzazione: la fase successiva alla raccolta dei dati consiste nello sviluppo di una IA addestrata in maniera non supervisionata che sia in grado di suddividere le varie immagini in \emph{cluster}, differenziati in base alle caratteristiche riconosciute in ogni sito. 
    \item IA classificativa: in seguito alla raccolta di un \emph{\gls{dataset}}\glsfirstoccur di dimensioni congrue si procede alla suddivisione manuale degli screenshot precedentemente clusterizzati su una base qualitativa (siti migliorabili e siti ottimi); questo processo viene svolto con l'ottica dell'addestramento di una IA classificativa.
    L'IA addestrata in maniera supervisionata ha l'obiettivo di affidare un punteggio in valori centesimali a ogni sito. 
    \item Invio di e-mail automatizzato: l'automazione della posta elettronica procede con l'invio di e-mail personalizzate ai proprietari dei siti web che hanno ricevuto una valutazione scarsa, per offrire loro un servizio di miglioramento.
\end{itemize}

\section{Analisi e gestione dei rischi}
Durante l'analisi del progetto lo stagista ha individuato alcuni rischi in cui potrà incorrere.
Nella lista seguente sono elencati i rischi e le soluzioni ideate.\\

\begin{risk}{Rimodulazione dell'attività}
    \riskdescription{dopo un mese dall’inizio dello stage lo studente è stato riposizionato sul progetto attuale scartando il progetto precedente e trovandosi dunque con meno settimane a disposizione per la produzione}
    \risksolution{ridimensionamento delle attività e richieste di supporto più frequenti}
    \label{risk: tempistiche ristrette} 
\end{risk}

\begin{risk}{Approccio sperimentale}
    \riskdescription{il progetto prevede dei contributi originali e sperimentali per cui non sono disponibili soluzioni già pronte}
    \risksolution{auto apprendimento tramite tutorial online e richiesta di coinvolgimento di colleghi più esperti nell'ambito}
    \label{risk:conoscenze scarse} 
\end{risk}

\begin{risk}{Costo dell'Addestramento}
    \riskdescription{l'addestramento dell'intelligenza artificiale richiede l'utilizzo di potenti GPU di cui spesso l'hardware a disposizione è sprovvisto}
    \risksolution{utilizzo del server aziendale per l’addestramento su CPU, ottenendo un compromesso tra costi e tempo di addestramento}
    \label{risk:hardware} 
\end{risk}

\begin{risk}{Quantità di dati di training insufficiente}
    \riskdescription{l'IA necessita una grande quantità di dati in input per effettuare un training efficace}
    \risksolution{aumento manuale del \emph{\gls{dataset}} e ricerca di \emph{\gls{dataset}} già pronti online}
    \label{risk:hardware} 
\end{risk}

%AGGIUNGERE ALTRI RISCHI versioni dipendenti da altre

\newpage

\begin{risk}{Risultati della clusterizzazione non soddisfacenti}
    \riskdescription{il \emph{clustering} potrebbe risultare non conforme alle aspettative}
    \risksolution{valutare la quantità di cluster da creare e sperimentare con altri metodi di \emph{clustering}}
    \label{risk:hardware} 
\end{risk}


\begin{risk}{Overfitting del modello}
    \riskdescription{il modello IA fornisce valutazioni accurate solo per le immagini utilizzate durante il training}
    \risksolution{sperimentare con metodi per la risoluzione dell'\emph{overfit (dropout, cross-validation, ecc...)}}
    \label{risk:hardware} 
\end{risk}

\begin{risk}{Dipendenze delle librerie}
    \riskdescription{il progetto utilizza molte tecnologie e librerie differenti, è probabile che si presentino delle incompatibilità}
    \risksolution{utilizzo del minor numero possibile di librerie e verifica delle compatibilità prima dell'inizio della scrittura del codice}
    \label{risk:hardware} 
\end{risk}

\section{Obiettivi}
Gli obiettivi hanno lo scopo di delineare il percorso che lo studente deve affrontare per portare a termine il progetto nella maniera desiderata dall'azienda.
Sono suddivisibili in:
\begin{itemize}
    \item \textbf{O}: obbligatori
    \item \textbf{D}: desiderabili
    \item \textbf{F}: facoltativi
\end{itemize}

\begin{table}[h!]
    \centering
    \begin{tabularx}{0.8\textwidth}{|c|X|}
    \hline
    \textbf{Codice} & \textbf{Descrizione}\\
    \hline
    O01 & Implementare un sistema robusto per la cattura degli screenshot, assicurando l’integrazione con il database per l’archiviazione e l’analisi. \\
    \hline
    O02 & Garantire la creazione di una documentazione tecnica completa che supporti sia l’uso che la manutenzione  del sistema sviluppato. \\
    \hline
    O03 & Creazione di un IA in grado di suddividere in cluster i siti web. \\
    \hline
    O04 & Creazione di un IA classificativa in grado di assegnare un punteggio ai siti web analizzati.\\
    \hline
    D01 & Aggiungere la valutazione creata dall'IA nel database utilizzato dal CRM.\\
    \hline
    D02 & Automatizzare l'invio di e-mail alle aziende che hanno ottenuto una valutazione scarsa.\\
    \hline
    F01 & Migliorare la raccolta dei dati delle aziende dal web\\
    \hline
    F02 & Aggiornare automaticamente la valutazione dei siti web\\
    \hline
    \end{tabularx}
    \caption{Tabella degli obiettivi}
    \end{table}

%AGGIUNGERE SE HO TROPPE POCHE PAGINE
%\section{Pianificazione}

