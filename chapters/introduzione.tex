\chapter{Introduzione}
\label{cap:introduzione}
\intro{L'esperienza dello stage è l'elemento che porta alla conclusione del percorso accademico dello studente; pertanto è da considerare una delle prove più formanti e impegnative di tutto l'arco di studi.\\
Durante questo periodo lo studente ha l'occasione di mettere alla prova le conoscenze acquisite negli anni, ma soprattutto ha la possibilità di valutare le proprie abilità di autoapprendimento, scontrandosi con tecnologie e problemi mai visti in precedenza. 
L'introduzione al mondo del lavoro è fondamentale per rendere il più lineare possibile la transizione da studio ad applicazione pratica delle conoscenze, fornisce inoltre un assaggio delle responsabilità che gravano su un programmatore durante la permanenza nella propria azienda.
In questo capitolo viene presentata l'azienda Spazio Dev presso la quale lo studente ha svolto lo stage nel periodo che va dal giorno 23/09/2024 al 14/11/2024.}

%
%\noindent Esempio di utilizzo di un termine nel glossario \\
%\gls{api}. \\
%
%\noindent Esempio di citazione in linea \\
%\cite{site:agile-manifesto}. \\
%
%\noindent Esempio di citazione nel pie' di pagina \\
%citazione\footcite{womak:lean-thinking} \\
%

\section{L'azienda}
Spazio Dev è una software house italiana nata a Tombolo nel 2023???; nonostante la giovane età è riuscita ad affermarsi nel suo campo dando vita a molteplici progetti ambiziosi. L'obiettivo primario dell'azienda è quello di supportare diverse imprese tramite soluzioni digitali e tecnologiche sviluppate su misura.
L'azienda conta 16 membri che ricoprono i diversi ruoli di sviluppatori, grafici, responsabili marketing e sistemisiti. L'organico ha un'età media inferiore ai trenta anni, questo fattore contribuisce a rendere l'ambiente lavorativo dinamico e conviviale perfetto per coltivare nuovi talenti.
Spazio Dev affianca e aggiorna il cliente tramite una comunicazione costante, finalizzata alla personalizzazione del prodotto in base alle diverse esigenze. Il focus principale dell'azienda è infatti quello di fornire prodotti specifici per ciascun cliente e non la vendita di prodotti standard.

\section{Servizi offerti}
L'offerta di Spazio Dev è molto ampia e consente allo staff di aiutare i clienti nei seguenti ambiti:
\begin{itemize}
  \item \textbf{Sviluppo di siti web}: La produzione di siti web è uno dei servizi più richiesti all'azienda e rappresenta uno dei motori trainanti. Gli sviluppatori hanno un occhio di riguardo per le nuove tecnologie, che sfruttano per offrire al cliente un prodotto accattivante e ottimizzato in maniera tale da garantire numerose visite.
  \item \textbf{Sviluppo software e integrazione IA}: I software prodotti sono solitamente indirizzati a una clientela variegata e possono spaziare in vari campi, in base alle esigenze di ciascun acquirente. Gli applicativi più venduti hanno come scopo principale la gestione dell'azienda e delle relazioni con i clienti. Per differenziarsi dalla concorrenza Spazio Dev ha introdotto nuove funzionalità impiegando modelli di IA capaci di predire trend futuri di acquisto e  di anticipare i bisogni dei clienti.
  \item \textbf{Marketing}: Gli esperti di marketing creano contenuti di alta qualità, pertinenti e ottimizzati, atti a migliorare la presenza on-line del brand. Per aumentare l'engagement vengono studiate apposite campagne social e di posta elettronica.
\end{itemize}

\section{Clientela}
La clientela di Spazio Dev è molto variegata ed è suddivisibile nelle seguenti categorie:
\begin{itemize}
  \item imprese di varie dimensioni specializzate in commercio e manifattura che necessitano di migliorare la propria presenza nel web, la gestione della contabilità, la logistica e le risorse umane.
  \item start-up in cerca di visibilità e di strategie di marketing.
  \item liberi professionisti come avvocati, medici, eccetera.
\end{itemize}


\section{Prodotto di punta}
RelAI rappresenta in tutto e per tutto l'azienda poiché nasce dall'unione di tutti i suoi core business: sviluppo web, integrazione IA e marketing.
Il software è un CRM (Cusotmer Relationship Manager) in cloud che consente di personalizzare le relazioni con i clienti e di incrementare le vendite grazie a campagne di comunicazione su misura.
La sua architettura modulare e le numerose configurazioni si adattano perfettamente in molti settori. L'IA implementata è utile per predire futuri trend di acquisto fondamentali per la creazione di strategie di marketing efficaci. 


\section{Organizzazione del testo}

\begin{description}
    \item[{\hyperref[cap:processi-metodologie]{Il secondo capitolo}}] descrive ...
    
    \item[{\hyperref[cap:descrizione-stage]{Il terzo capitolo}}] approfondisce ...
    
    \item[{\hyperref[cap:analisi-requisiti]{Il quarto capitolo}}] approfondisce ...
    
    \item[{\hyperref[cap:progettazione-codifica]{Il quinto capitolo}}] approfondisce ...
    
    \item[{\hyperref[cap:verifica-validazione]{Il sesto capitolo}}] approfondisce ...
    
    \item[{\hyperref[cap:conclusioni]{Nel settimo capitolo}}] descrive ...
\end{description}

Per quanto riguarda la stesura del documento sono state adottate le seguenti convenzioni tipografiche:
\begin{itemize}
	\item gli acronimi, le abbreviazioni e i termini ambigui o di uso non comune menzionati vengono definiti nel glossario, situato alla fine del presente documento;
	\item per la prima occorrenza dei termini riportati nel glossario viene utilizzata la seguente nomenclatura: \emph{parola}\glsfirstoccur;
	\item i termini in lingua straniera o facenti parti del gergo tecnico sono evidenziati con il carattere \emph{corsivo}.
\end{itemize}
