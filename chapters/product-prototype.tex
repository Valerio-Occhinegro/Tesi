\chapter{Strumenti e tecnologie}
\label{cap:strumenti-tecnologie}

\intro{L'obiettivo principale di questo capitolo è l'illustrazione delle tecnologiene degli strumenti ausiliari utilizzati per raggiungere lo scopo finale del progetto}\\

\section{Strumenti}
\label{sec:strumenti}
Gli strumenti di supporto sfruttati per il progetto sono elencati nella lista seguente.

\begin{itemize}
    \item Visual Studio Code è un ambiente di sviluppo integrato, disponibile per Linux, macOS e Windows. 
    È un'applicazione che supporta la maggior parte dei linguaggi di programmazione ed è quindi molto vantaggiosa per lavorare su un progetto multi-linguaggio senza dover cambiare ambiente.
    Un'altra delle funzionalità principali è la fornitura di numerose estensioni che semplificano il processo di scrittura e verifica del codice.

    \begin{figure}[!h] 
        \centering 
        \includegraphics[width=0.4\columnwidth]{tecnologie/visual-studio-logo.png} 
        \caption{Logo di Visual Studio Code}
      \end{figure}

\newpage

    \item PhpMyAdmin è una web-app scritta utilizzando il linguaggio di programmazione PHP, che offre la capacità di gestione di un database MySQL attraverso un browser qualsiasi. Consente la creazione di tabelle, l'inserimento, la modifica e l'interrogazione dei dati.
    Fornisce un'interfaccia grafica per la visione d'insieme e per le operazioni amministrative.

    \begin{figure}[!h] 
        \centering 
        \includegraphics[width=0.4\columnwidth]{tecnologie/phpmyadmin-logo.png} 
        \caption{Logo di phpMyAdmin}
      \end{figure}

    
    \item Remote Ripple è un software per l'accesso remoto, che viene utilizzato dallo studente per avviare i programmi presenti nel server aziendale.
    
    \begin{figure}[!h] 
        \centering 
        \includegraphics[width=0.3\columnwidth]{tecnologie/remote-ripple-logo.png} 
        \caption{Logo di Remote Ripple}
      \end{figure}

\end{itemize}

\section{Tecnologie}
\label{sec:tecnologie-strumenti}

Di seguito viene data una panoramica delle tecnologie e strumenti utilizzati.

\subsection*{Tecnologia 1}
Descrizione Tecnologia 1.

\subsection*{Tecnologia 2}
Descrizione Tecnologia 2

\section{Ciclo di vita del software}
\label{sec:ciclo-vita-software}

\section{Progettazione}
\label{sec:progettazione}

\subsubsection{Namespace 1} %**************************
Descrizione namespace 1.

\begin{namespacedesc}
    \classdesc{Classe 1}{Descrizione classe 1}
    \classdesc{Classe 2}{Descrizione classe 2}
\end{namespacedesc}


\section{Design Pattern utilizzati}

\section{Codifica}
