\cleardoublepage
\phantomsection
\pdfbookmark{Sommario}{Sommario}
\begingroup
\let\clearpage\relax
\let\cleardoublepage\relax
\let\cleardoublepage\relax

\chapter*{Sommario}

Il presente documento descrive il lavoro svolto durante il periodo di stage dal laureando
Valerio Occhinegro presso l’azienda Spazio Dev Srl di Tombolo (PD). Lo stage, svoltosi al termine del percorso di studi della Laurea Triennale in "Scienze Informatiche", ha avuto una durata complessiva di 306 ore.

Il lavoro in questione è stato suddiviso in 4 diverse fasi, ognuna delle quali caratterizzata da obiettivi specifici e scadenze precise, e si basa su un'analisi intelligente di siti web, improntata alla vendita dei servizi aziendali a potenziali clienti. Il ruolo del laureando è quello di gestire e archiviare i siti web target e successivamente classificarli per fornire a Spazio Dev input utiili per acquisire l'abilità di ampliare la quantità dei propri clienti in maniera rapida e automatizzata. 
Nello specifico, il progetto di stage ha l’obiettivo di sviluppare competenze avanzate nell’intelligenza artificiale, con un focus
particolare sulla classificazione e organizzazione dei dati; uno studio che si inserisce nel contesto di
{“SalesCRM: Customer Relationship Manager ”, un sistema integrato che mira a ottimizzare la gestione delle relazioni con i clienti e a migliorare l’efficienza dei venditori.} INSERIRE VERO TITOLO
Lo scopo ultimo del progetto è la creazione di un sistema intelligente che sia in grado di analizzare dataset complessi,
classificare la qualità dei siti web (siti che potrebbero essere migliorati e siti che non necessitano di modifiche) e proporre i propri servizi alle aziende che ne necessitano in maniera automatica. 
Il laureando è responsabile dello sviluppo di funzionalità chiave del sistema che consentiranno di fornire
soluzioni strategiche agli utilizzatori della piattaforma. 



%\vfill

%\selectlanguage{english}
%\pdfbookmark{Abstract}{Abstract}
%\chapter*{Abstract}

%\selectlanguage{italian}

\endgroup

\vfill
