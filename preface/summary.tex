\cleardoublepage
\phantomsection
\pdfbookmark{Sommario}{Sommario}
\begingroup
\let\clearpage\relax
\let\cleardoublepage\relax
\let\cleardoublepage\relax

\chapter*{Sommario}

Il presente documento descrive il lavoro svolto durante il periodo di stage dal laureando
Valerio Occhinegro presso l’azienda Spazio Dev Srl di Tombolo (PD). Lo stage, svoltosi al termine del percorso di studi della Laurea Triennale in "Scienze Informatiche", ha avuto una durata complessiva di 306 ore.

Il lavoro in questione è stato suddiviso in 4 diverse fasi, ciascuna caratterizzata da obiettivi specifici e scadenze precise, e si basa sul monitoraggio intelligente e l'ottimizzazione della salute delle colture tramite l'utilizzo di tecnologie di sensoristica avanzata e analisi dati. Il ruolo del laureando è quello di gestire il flusso di dati ambientali e agronomici per generali insight predittivi sulla gestione delle colture, partecipando al contempo allo sviluppo di soluzioni tecnologiche che possano aiutare gli agricoltori a prendere decisioni basate su dati concreti. 
Nello specifico, il progetto di stage ha l’obiettivo di sviluppare competenze avanzate nell’intelligenza artificiale, con un focus
particolare sulla classificazione e organizzazione dei dati in ambito agricolo; uno studio che si inserisce nel contesto di
“SmartFarm Technologies: L’Agricoltura del Futuro”, un sistema integrato che mira a ottimizzare la gestione agricola,
migliorare l’efficienza delle risorse e aumentare la sostenibilità.
Lo scopo ultimo del progetto è la creazione di un sistema intelligente che sia in grado di analizzare dataset agricoli complessi,
classificare le diverse condizioni delle colture (come malattie, carenze nutrizionali, o stress idrico) e proporre soluzioni
mirate per il miglioramento delle stesse. Queste soluzioni potranno includere raccomandazioni per l’irrigazione,
l’applicazione di fertilizzanti o pesticidi o altre pratiche agricole ottimizzate per le condizioni specifiche rilevate.
Il laureando è responsabile dello sviluppo di funzionalità chiave del sistema che consentiranno di fornire
raccomandazioni strategiche e soluzioni pratiche agli utilizzatori della piattaforma, come agricoltori e tecnici agricoli, pur non essendo questi ultimi i clienti finali. 



%\vfill

%\selectlanguage{english}
%\pdfbookmark{Abstract}{Abstract}
%\chapter*{Abstract}

%\selectlanguage{italian}

\endgroup

\vfill
