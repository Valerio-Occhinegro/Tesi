

% Glossary entries
%\newglossaryentry{apig} {
%    name=\glslink{api}{API},
%    text=Application Program Interface,
%    sort=api,
%    description={in informatica con il termine \emph{Application Programming Interface API} (ing. interfaccia di programmazione %di un'applicazione) si indica ogni insieme di procedure disponibili al programmatore, di solito raggruppate a formare un set %di strumenti specifici per l'espletamento di un determinato compito all'interno di un certo programma. La finalità è %ottenere un'astrazione, di solito tra l'hardware e il programmatore o tra software a basso e quello ad alto livello %semplificando così il lavoro di programmazione}
%}
%
%\newglossaryentry{umlg} {
%    name=\glslink{uml}{UML},
%    text=UML,
%    sort=uml,
%    description={in ingegneria del software \emph{UML, Unified Modeling Language} (ing. linguaggio di modellazione unificato) è %un linguaggio di modellazione e specifica basato sul paradigma object-oriented. L'\emph{UML} svolge un'importantissima %funzione di ``lingua franca'' nella comunità della progettazione e programmazione a oggetti. Gran parte della letteratura di %settore usa tale linguaggio per descrivere soluzioni analitiche e progettuali in modo sintetico e comprensibile a un vasto %pubblico}
%}

\newglossaryentry{CRM} {
    name=CRM,
    text=CRM,
    sort=crm,
    description={Un software di CRM si occupa di gestire il marketing, il supporto e le vendite dell'azienda in maniera automatizzata. Più in generale è l'insieme di strategie utilizzate per migliorare e organizzare tutti i rapporti con i clienti}
}

\newglossaryentry{UML} {
    name=UML,
    text=UML,
    sort=uml,
    description={L'unified modelling language è un linguaggio di modellazione unificato nato per semplificare la comunicazione di concetti complicati, tramite l'utilizzo di schemi e immagini codificate}
}

\newglossaryentry{open source} {
    name=open source,
    text=open source,
    sort=open source,
    description={L'open source è una tipologia di software di libera distribuzione, che può essere utilizzato, modificato e redistribuito da chiunque a titolo gratuito}
}

\newglossaryentry{bot} {
    name=bot,
    text=bot,
    sort=bot,
    description={I bot sono dei programmi inseriti all'interno di telegram per eseguire determinate azioni in maniera automatica}
}

\newglossaryentry{self hosted} {
    name=self hosted,
    text=self hosted,
    sort=self hosted,
    description={Con self hosted si intende l'esecuzione del software all'interno di server aziendali}
}

\newglossaryentry{workflow} {
    name=workflow,
    text=workflow,
    sort=workflow,
    description={Il termine workflow o flusso di lavoro indica la struttura che si occupa di gestire un insieme di processi da svolgere in un ordine predefinito}
}

\newglossaryentry{web scraping} {
    name=web scraping,
    text=web scraping,
    sort=web scraping,
    description={Il web scraping si occupa dell'estrazione di informazioni da siti web tramite strumenti automatici}
}

\newglossaryentry{dataset} {
    name=dataset,
    text=dataset,
    sort=dataset,
    description={Un dataset è una raccolta di dati appartenenti a determinate categorie}
}

\newglossaryentry{feature} {
    name=feature,
    text=feature,
    sort=feature,
    description={Le feature sono le caratteristiche che vengono utilizzate per addestrare un algoritmo di machine learning}
}

\newglossaryentry{web app} {
    name=web app,
    text=web app,
    sort=web app,
    description={È un'applicazione utilizzabile tramite il web}
}

\newglossaryentry{framework} {
    name=framework,
    text=framework,
    sort=framework,
    description={Un framework è un sistema che permette di ampliare le capacità di un linguaggio di programmazione, semplificando e ottimizzando lo svolgimento di determinate azioni da parte del programmatore}
}

\newglossaryentry{API} {
    name=API,
    text=API,
    sort=API,
    description={Le API sono un insieme di regole utilizzate per consentire a software diversi di comunicare tra loro}
}

\newglossaryentry{computer vision} {
    name=computer vision,
    text=computer vision,
    sort=computer vision,
    description={La computer vision si occupa di sviluppare algoritmi in grado di estrarre informazioni da immagini, consentendo così al computer di "vedere"}
}

\newglossaryentry{V-RAM} {
    name=V-RAM,
    text=V-RAM,
    sort=V-RAM,
    description={È la memoria contenuta all'interno delle schede video, è una componente fondamentale per consentire alla GPU di processare informazioni}
}

